%文档类型
\documentclass[a4paper,twocolumn,landscape]{article}

%引用包裹
\usepackage{bm}
\usepackage{cmap}
\usepackage{ctex}
\usepackage{cite}
\usepackage{color}
\usepackage{float}
\usepackage{xeCJK}
\usepackage{amsthm}
\usepackage{amsmath}
\usepackage{amssymb}
\usepackage{setspace}
\usepackage{geometry}
\usepackage{hyperref}
\usepackage{enumerate}
\usepackage{indentfirst}
\usepackage[cache=false]{minted}
\usepackage{fontspec}
\usepackage{pdfpages}
\usepackage{fancyhdr}
\usepackage[table]{xcolor}
\usepackage{booktabs}
\usepackage{harpoon}
%代码高亮
\geometry{margin=1in}
\setmonofont{Ubuntu Mono}
%字体设置
\setmainfont{Ubuntu Mono}
%\setCJKmonofont{SimSun}
%\setCJKmainfont[BoldFont={SimSun}]{SimSun}
%\pagestyle{fancy}
\hypersetup{
	colorlinks=true,
	linkcolor=black
}

%\newcommand{\cppcode}[1]{
 %   \inputminted[mathescape,
  %  frame=lines,linenos]{cpp}{source/#1}
%}

%\newcommand{\javacode}[1]{
 %   \inputminted[mathescape,
  %  frame=lines,linenos]{java}{source/#1}
%}

\newcommand{\cppcode}[1]{
	\inputminted[mathescape,
	tabsize=2,
	style=xcode,
	%linenos,
	%frame=single,
	autogobble,
	framesep=0.5mm,
	breakaftergroup=true,
	breakautoindent=true,
	breakbytoken=true,
	breaklines=true,
	fontsize=\small
	]{cpp}{source/#1}
}
\newcommand{\javacode}[1]{
	\inputminted[mathescape,
	tabsize=2,
	%linenos,
	%frame=single,
	framesep=2mm,
	breakaftergroup=true,
	breakautoindent=true,
	breakbytoken=true,
	breaklines=true,
	fontsize=\small
	]{java}{source/#1}
}

\newcommand{\vimcode}[1]{
	\inputminted[mathescape,
	tabsize=2,
	%linenos,
	%frame=single,
	framesep=2mm,
	breakaftergroup=true,
	breakautoindent=true,
	breakbytoken=true,
	breaklines=true,
	fontsize=\small
	]{vim}{source/#1}
}

\begin{document}
\iffalse

\title{代码库}
\author{Blazar}
\date{\today}

\newpage
\newpage

\maketitle

\tableofcontents

\newpage

\fi


\section{字符串}

\subsection{KMP算法}

\cppcode{temp_ypm/KMP-poj3461.cpp}

\subsection{扩展KMP算法}

%返回结果:$$next_i = lcp(text, text_{i \dots n-1})$$

%\cppcode{string-manipulation/ExtKMP.cpp}
\cppcode{temp_ypm/exKMP-hdu2594.cpp}

\subsection{AC自动机}

%\cppcode{string-manipulation/ACmachine.cpp}
\cppcode{temp_ypm/acauto.cpp}
\subsection{后缀自动机}

\subsubsection{广义后缀自动机(多串)}
\noindent \textbf{注意事项:}空间是插入字符串总长度的2倍并请注意字符集大小。
\cppcode{temp_ypm/gy sam.cpp}

\subsubsection{sam-ypm}
\subsubsection*{sam-nsubstr}
\cppcode{temp_ypm/SAM-NSUBSTR.cpp}
\subsubsection*{sam-lcs}
\cppcode{temp_ypm/SAM-LCS2.cpp}

\subsection{后缀数组}
\noindent \textbf{注意事项:}$\mathcal{O}(n\log n)$倍增构造。
%\cppcode{String-Algorithm/Suffix-Array.cpp}
\cppcode{temp_ypm/SA.cpp}

%\noindent \textbf{注意:}$\mathcal{O}(n)$线性构造,常数大,约为倍增的0.5-0.6倍
%\cppcode{temp_ypm/dc3-uoj34.cpp}

%\subsection{回文自动机}
%\noindent \textbf{注意事项:}请注意字符集大小。
%\cppcode{String-Algorithm/Palindromic-Automaton.cpp}
%\cppcode{temp_ypm/PAM.cpp}
\subsection{Manacher}
\noindent \textbf{注意事项:}1-based算法,请注意下标。
\cppcode{temp_ypm/manacher-poj3974.cpp}
\subsection{循环串的最小表示}
\noindent \textbf{注意事项:}0-Based算法,请注意下标。
%\cppcode{String-Algorithm/Minimum-Representation.cpp}
\cppcode{temp_ypm/minimumpresentation.cpp}

\iffalse
\subsection{后缀树}
\noindent \textbf{注意事项:}
\begin{enumerate}
	\item \indent 边上的字符区间是左闭右开区间;
	\item \indent 如果要建立关于多个串的后缀树,请用不同的分隔符,并且对于每个叶子结点,去掉和它父亲的连边上出现的第一个分隔符之后的所有字符;
\end{enumerate}
\cppcode{String-Algorithm/Suffix-Tree.cpp}
\fi

\end{document}
